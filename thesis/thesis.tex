\documentclass{xduugthesis}
\xdusetup{style = {after-skip = {18pt,12pt,12pt,12pt,10pt,10pt}, before-skip = {20pt,14pt,14pt,14pt,12pt,12pt}}}
\begin{document}
\chapter{概述}
\section{研究背景和意义}
合成孔径雷达(Synthetic Aperture Radar, SAR)是一种利用雷达信号的相位和幅度信息来生成图像的技术。它通过移动的雷达平台发射和接收信号,然后对这些信号进行处理,以产生高分辨率的图像。SAR技术在军事、地质勘探、环境监测、农业和城市规划等多个领域有着广泛的应用。\par
在实际应用中,尤其是对于遥感图像进行目标识别的情况下,使用者需要对大量的图像进行融合操作。在这个过程中,不同的图像融合方法都会对最终图像的效果产生不同的影响。在这个方面,本毕设将会调用实现一些常见的图像融合算法,并进行评估,从而加深对这些图像融合算法的认识。\par
同时,由于目前公开的合成孔径雷达数据集比较少,尤其缺少对一个目标点进行多角度成像的图片数据集。本毕设将会实现一个对雷达成像过程的仿真,并基于此进行包装,编写一个仿真生成合成孔径雷达图像特征图像程序。\par
最终,为了方便最终用户使用,我们需要将上述提到的融合算法和仿真程序进行包装,并进行用户界面包装,从而让用户更加直接地使用这些功能。\par
本合成孔径雷达图像融合程序可以使用户更加容易地进行雷达图像融合。同时,通过对图像融合算法的评估,用户可以更加直观地了解到这些算法的特点,从而挑选更加合适的算法,或者进一步优化。程序里面的仿真模块也是入门雷达图像扫描过程的直观参考。\par
\section{研究内容}
本毕设主要解决三个问题:针对合成孔径雷达图像特征进行图像融合,仿真生成合成孔径雷达图像特征的图像,根据用户界面优化成为一个程序。\par
在针对合成孔径雷达图像特征进行图像融合方面,本毕设需要在了解图像分解和融合原理的基础上,引用和实现一些常见的图像分解融合方法。同时,我们需要在了解图像评估方法对基础上,对各个方法融合的图像进行评估。\par
在仿真生成合成孔径雷达图像特征的图像方面,本毕设需要寻找合适的工具,进行对合成孔径雷达扫描目标点,形成图像过程的模拟。进而,获得对一个目标点进行多角度的图像,方便用户在没有数据的情况下,使用这个软件,进而对合成孔径雷达成像和图像融合结果形成一个初步的认知。\par
最后,在根据用户界面优化成为一个程序方面,我们需要将这些功能的交互进行优化,从而使用户更直观地使用本程序。
\section{论文结构}
本论文共分为五章,具体结构如下:\par
第一章是概述。主要是为了提出本毕设需要解决的问题,以及本毕设一个大致的架构。\par
第二章介绍了相关技术。本章主要介绍本毕设所涉及的主要背景,包括雷达简介,图像融合技术和仿真使用的软件。\par
第三章介绍了系统的设计和实现。根据第一章提到的毕设大致架构,阐述了本毕设的实现方法和细节,以及如何让用户更加方便地使用本软件。\par
第四章介绍了本软件运行的状况。本章模拟了两个用户使用场景,展示了本软件使用指南,同时展示了图像融合算法的处理结果。\par
第五章对文章进行总结和展望,总结所做的工作并给出未来的研究方向和展望。
\chapter{相关技术背景}
\section{合成孔径雷达}
本段将会对合成孔径雷达进行简单介绍,同时介绍其成像步骤。
\subsection{简介}
合成孔径雷达(英语:synthetic aperture radar, SAR),属于一种微波成像雷达,也是一种可以产生高分辨率图像的(航空)机载雷达或(太空)星载雷达。它通过移动的雷达平台发射和接收电磁波信号,然后对这些信号进行处理,以产生高分辨率的图像。使用SAR雷达成像,相比使用光学设备,能够穿透云层和雾层笼罩,进而较不受天气影响。与此同时,由于利用了电磁波反射的特性,SAR雷达成像能够体现出地貌或被观测物体的特征。目前,该技术在军对地观测、行星观测、目标定位、无人驾驶等领域中有着广泛的应用。\par
根据种类,合成孔径雷达可以形成可以分成单发单收、单发多收等系统。这些是根据在一个雷达系统中,发出信号的设备和接收设备的多少来进行分类的。在目标定位中,使用多个信号源能够提高成像的信号量,从而提高定位精度。本论文只讨论单发单收的状况。
\subsection{成像过程}
根据刘国祥的介绍[刘国祥],我在此简要描述单发单收合成孔径雷达的成像过程。\par
我们假设雷达的天线宽度是W,长度是L。雷达按照轨道,在高度H上运行。雷达以一个侧视角$\theta_0$发射一个椭圆锥状的微波脉冲,该脉冲垂直于轨道飞行方向。其垂直于轨道的顶角称为波束高度角$\omega_v$,它与雷达天线宽度是W和雷达波长$\lambda$相关,即:
$$\omega_v=\frac{\lambda}{W}$$
其沿轨的椭圆锥顶角$\omega_h$则与雷达天线长度L相关,即:\par
$$\omega_h=\frac{\lambda}{L}$$
该微波脉冲将会在地面形成一个辐照带。由上可知,雷达天线越宽,辐射带照得越远;天线越长,辐射带照得越宽。\par
辐照带的辐宽是垂直于轨道方向的辐照带长度,计算方式如下:\par
$$W_G\approx\frac{{\lambda}R_m}{wcos\eta}$$
其中,$R_m$为雷达中心到辐照带中心的斜距,$\eta$为该中心点的雷达入射角。\par
评估SAR雷达成像的信息量和效果称为分辨率,也就是可以区分两个相邻目标的最小距离。分辨率越小越好。分辨率有两个方向的,称为方位角分辨率$\Delta X$和斜距向分辨率$\Delta R$。\par
其中,方位向分辨率按照雷达斜距R和雷达长度确定,计算公式为:\par
$$\Delta\ X=R\omega_h=\frac{R\lambda}{L}$$
斜距向分辨率投射到地面时候,是斜距向地面分辨率$\Delta Y$。这两个分辨率和光速c,脉冲宽度$\tau_p$和侧视角$\theta_i$相关,计算公式为:\par
$$\Delta R=\frac{c\tau_p}{2}$$
$$\Delta Y=\frac{\Delta R}{cos(90^\circ-\theta_i)}=\frac{c\tau_p}{2cos(90^\circ-\theta_i)}$$
下图展示了SAR雷达的一次扫描形成的辐照带和分辨率。\par
在实际应用中,我们需要针对一个目标点不停成像,从而提高目标的分辨率。斜距向分辨率的参数只和雷达性能参数相关,改善的话不能从此着手。我们需要利用多普勒频移现象来改善雷达成像的方位向分辨率。假设长度为L的雷达天线从a移动到b再到c(其中b为ac之间的中点),在扫描目标过程中,我们需要测定脉冲的延迟,跟踪频率转移。最后合成一个脉冲,提高目标成像分辨率。此时方位向分辨率$\Delta X$可以近似表述为:\par
$$\Delta X=\frac{L}{2}$$
\section{图像融合算法}
本段介绍图像融合算法的大致步骤,本毕设涉及的图像融合算法。
\subsection{原理和意义}
图像融合过程是指从多幅图像中收集所有重要信息,并将其合并成较少的图像,通常是单幅图像。这种单一图像比任何单一来源的图像信息量更大、更准确,而且包含所有必要的信息。图像融合的目的不仅在于减少数据量,还在于构建更适合人类和机器感知、更易于理解的图像。图像融合过程分为图片分解方法和图像融合方法,而最终的算法是由一个分解方法和一个合成方法组成的。\par
在本毕设中,我们需要将多张对单个目标多角度扫描所形成对一系列图像进行合成,最后形成一张最大程度包括这些成像信息的图像,提高该图像的信息量。
\subsection{图像分解方法}
图像分解方法是对同一个目标一系列图像的特征进行分解,从而获取出图像的一系列特征。以下介绍常见的图像分解算法。\par
\subsubsection{小波图像分解方法}
小波图像分解【介绍来源】基于一个限制时域的,会衰减的小波基来对图像进行处理。小波可以进行在时间上的拉伸和收缩过程,以及在信号延迟上的提前和错后。小波变换相比经常使用的傅立叶变换,能够知道各个成分出现的时间,知道信号频率随时间变化的情况。这个应用到图像,可以做到让图像每个频谱的时空信息都能保留,并能对其进行分析。在本毕设中,我将使用haar小波来进行小波变换。\par
下图展示了对图像使用小波算法分解后,会得到图像的低频信息,水平方向的高频信息,垂直方向的高频信息和对角线的高频信息。同时,我们可以对低频信息继续分解,进行多次图像分解。\par
小波图像分解融合算法实现在很多重要的图形处理库都有,比如OpenCV,MATLAB的图像处理包等。\par

接下来四张图像分别表示一张图片一次小波分解后的低频信息,水平方向、竖直方向和对角线方向的高频信息。\par

\subsubsection{NSCT图像分解方法}
NSCT图像分解【介绍来源】基于金字塔分解滤波器和非下采样方向滤波器,对图像进行高低频和方向面的分解。首先,使用金字塔分解滤波器多次分解图像,得到低频和高频信息。然后,对分解后的高频分量进行方向分解,分解成为不同方向上的细节信息。\par

NSCT图像分解方法相比小波图像分解方法,可以更加精细地分解图像,更好地保留出高频信息(比如图像中的线,边缘等信息),同时避免了信号变化极大的时候容易出现的振荡现象。\par
NSCT图像处理算法成熟的实现目前只有随该分解方法论文附带的MATLAB处理库【】。\par

\subsection{图像融合方法}
图像融合方法是针对上述提到分解成果,选择适当的方式提取图像特征,进而进行融合。最后,进行上述图像分解方法的逆向方法,最后成为一张融合后的图像。\par
\subsubsection{最大绝对值融合法}
最大绝对值融合法是逐像素来进行融合的方法。目标是将每个图像中,最突出的信息集中到最终融合图像中,从而,提高最终图像的信息量。像素信息的突出方式,也就是该像素的亮度,在黑白影像中,一般按照八位的无符号整数来表示。\par
本方法需要对每张需要融合图片的同一位置像素亮度进行分析,以绝对值最大者为最终融合图像对应位置的的像素。公式如下:
$$w_{dk}^{i,j}=max\left(\{\left|w_{dn}^{i,j}\right|,n\in\left[1,K\right],n\in N^\ast\}\right)$$
其中,$K$表示输入的图像数量。
\subsubsection{主成分变换融合法}
主成分变换(principal component analysis, PCA)是一个常用的数据降维算法,通过这个算法,可以提取出数据的特征。\par
根据这篇论文【】,我们可以将分解方法生成的低频信息进行主成分变换,然后将第一个分量作为新的近似图像。之后,将近似低频信号和高频信号分别按照以下公式进行加权:
$$HH(x,y)=\sum_{i=1}^{N}k_iHH_i(x,y)$$
$$k_i=\mu_i/\sum_{i=1}^{N}\mu_i$$
其中,$\mu_i$表示第$i$个图像的数据平均值,$HH_i(x,y)$表示需要融合的第$i$个图像,$HH(x,y)$表示融合后的图像。
\subsubsection{核范式加权融合法}
图像的核范式,是其我们可以通过对图像进行SVD奇异值分解,然后通过累加矩阵奇异值,获得核范式。\par
本方法需要对每张图片的奇异值进行分析,然后整体进行权重加权融合,得出最终融合的图像。该方法的权重公式【】如下:
$$w_{dk}^{i,j}=\frac{w_{dk}^{\hat{i},j}}{\sum_{q=1}^{K}w_{dq}^{\hat{i},j}}$$
$$w_{dk}^{\hat{i},j}={\left\| re\left(V_{dk}^{i,j}\right)\right\|}_*$$
其中,$re\left(V_{dk}^{i,j}\right)$表示需要重构的图像,$\cdot_*$表示该图像的核范式,$K$表示输入的图像数量。\par
\subsection{融合结果评估指标}
对于融合后的图像,只凭人的主管判断是不够的,使用一些参数进行量化评估是很有必要的。\par
\subsubsection{熵值}
图片的熵值(entropy)代表了图片所包含的信息量。具体定义为:
$$EN=-\sum_{i=0}^{L-1}p_ilog_2p_i$$
其中$L$代表灰度级数,$p_i$代表图片灰度级的直方图。其值越大,表示图片包含的信息越多。\par
\subsubsection{标准差}
图片的标准差(standard deviation, SD)表示图片信息的离散程度,其公式如下:
$$SD=\sqrt{\sum_{i=1}^{M}\sum_{j=1}^{N}(F(i,j)-\mu)^2}$$
其中$\mu$表示图片的均值。其值越大,表示数据离散程度越高,图片越能突出表示特征。
\subsubsection{互信息}
融合后的图片和融合前图片的互信息,表示融合前后图像信息的相关性,该值越大越能表示相关性越强。
\subsubsection{空间频率}
空间频率(spatial frequency,SF) 是通过测量融合图像的梯度分布,揭示融合图像的细节和纹理信息的指标。具体定义为
$$SF = RF^2 + CF^2$$
式中, $RF = \sqrt{\sum_{i=1}^{M}\sum_{j=1}^{N}(F(i,j)-F(i,j-1))^2}$表示行频率,$CF = \sqrt{\sum_{i=1}^{M}\sum_{j=1}^{N}(F(i,j)-F(i-1,j))^2}$表示列频率。更高的空间频率意味着更加丰富的边缘和纹理细节。
\section{仿真生成雷达图像}
本段介绍仿真生成雷达图像的必要性,以及本毕设涉及到的MATLAB雷达仿真工具箱。
\subsection{该过程的必要性}
本毕设需要对地面上的目标点进行多角度扫描后的图像数据。而在这个方面,目前关于SAR雷达成像数据比较少,其中最著名的是欧洲的ERS-1数据【】和美国军方的MSTAR数据【】。欧洲的ERS-1数据是很清晰的遥测数据,但是,这些数据缺少对一个目标多角度的扫描成图,而只是将图片旋转后裁剪,和显示成像结果。而MSTAR虽然包含了对一个目标多角度的扫描,但是目标过小,不利于图像的融合。因此,自行生成多角度的扫描图对于本毕设很有必要。\par
【插入点这俩玩意图片】\par
目前开源的合成孔径雷达软件只有RaySAR【】用户比较多,但是这个软件用途是分析建筑物结构的。虽然使用了合成孔径雷达模拟,但是不符合本毕设的用途。最后,本毕设决定使用MATLAB提供的雷达工具箱来进行仿真。
\subsection{MATLAB雷达仿真工具箱}
MATLAB雷达仿真工具是在2021年由美国MathWorks公司发布的雷达仿真工具,其中包括了设计、仿真、分析和测试多功能雷达系统的算法和工具。通过该工具,用户可以设计、模拟、分析和测试多功能雷达系统,进而加速项目的验证和落地。\par
目前,该工具在汽车雷达仿真、合成孔径雷达仿真、以及机载雷达设计方面得到了广泛应用。
\chapter{系统设计与实现}
\section{图像融合模块}
\subsection{融合算法实现}
通过对图像融合方法的分析和比较,我将实现以下算法:
\subsubsection{一次小波分解,最大绝对值融合法}
小波变换作为最广泛使用的图像分解算法,我将其作为分解图片的参照组。同时,最大绝对值融合法的编程和运算相对简单,于是将其作为融合时候的参照组。\par
本方法步骤如下:
\begin{enumerate}
	\item 对于输入的一系列图片,各进行一次小波分解。得到图片的低频信息和水平方向、竖直方向和对角线方向的高频信息。
	\item 对于每一个分解信息,进行最大绝对值融合法图像融合。
	\item 对于融合后的信息,使用小波变换的逆变换,完成融合步骤。
\end{enumerate}
\subsubsection{一次小波分解,主成分变换融合法}
这个方法是参考2.2.3.2方法实现的。由于小波多层分解中,下一层的分解利用了上一次分解的低频信息,如果输入图片过小,则造成多层分解后,最底层的低频信息主成分过少,从而没必要进行降低主成分操作。所以,本方法仅会进行一次小波分解。\par
本方法步骤如下:
\begin{enumerate}
	\item 对于输入的一系列图片,各进行一次小波分解。得到图片的低频信息和水平方向、竖直方向和对角线方向的高频信息。
	\item 对于图片的低频,使用主成分变换法,获取其第一个主成分分量。然后,分别计算这些信息的平均值,从而获得融合权重。
	\item 根据权重,对于每层图片的每一个分解信息,根据上一个步骤的权重进行加权平均融合。
	\item 对于融合后的信息,使用小波变换的逆变换,完成融合步骤。
\end{enumerate}
\subsubsection{三次小波分解,最大绝对值融合法}
如上个方法所述,小波变换的下一层分解是基于上一次分解的低频信息。所以,如果我们能多层进行分解,并在每层进行融合,也许会对融合后图像的质量提升有很大帮助。\par
本方法步骤如下:
\begin{enumerate}
	\item 对于输入的一系列图片,各进行三次小波分解。得到图片的低频信息和水平方向、竖直方向和对角线方向的高频信息。
	\item 对于分解后的信息,使用最大绝对值融合法进行融合。
	\item 对于融合后的信息,从底层到高层,使用小波变换的逆变换,完成融合步骤。
\end{enumerate}
\subsubsection{一次NSCT分解,最大绝对值融合法}
这个方法使用NSCT金字塔式分解,用于和纯小波变换进行比较。本方法步骤如下:
\begin{enumerate}
	\item 对于输入的一系列图片,各进行一次NSCT分解。得到图片的低频信息和高频信息。
	\item 对于分解后的信息,低频使用核范式加权融合法进行融合,对于高频信息,使用最大绝对值融合法进行融合。
	\item 对于融合后的信息,从底层到高层,使用小波变换的逆变换,完成融合步骤。
\end{enumerate}
\subsubsection{三次NSCT分解,主成分变换融合法}
NSCT的层级分解其只会生成一个低频信息,因为它的层级分解是针对高频信息的。所以,我们可以对NSCT三次分解,然后都使用主成分融合法进行加权融合。\par
本方法步骤如下:
\begin{enumerate}
	\item 对于输入的一系列图片,各进行三次NSCT分解。得到图片的低频信息和高频信息。
	\item 对于分解后的低频信息,使用主成分变换法,获取其第一个主成分分量。然后,计算这些信息的平均值,获得融合权重。
	\item 使用融合权重进行分解后信息的加权融合。
	\item 对于融合后的信息,使用NSCT变换的逆变换,完成融合步骤。
\end{enumerate}
\subsubsection{三次NSCT分解,核范式加权融合法}
步骤如下:
\begin{enumerate}
	\item 对于输入的一系列图片,各进行三次NSCT分解。得到图片的低频信息和三层高频信息。
	\item 对于分解后的信息,低频信息使用核范式加权融合法,各层高频信息使用最大绝对值融合法。
	\item 对于融合后的信息,从底层到高层使用三次NSCT变换的逆变换,完成融合步骤。
\end{enumerate}


\subsection{评估融合算法}
本毕设将会使用五张来自MSTAR数据集的坦克当作测试数据,这些数据都附带了角度,方便不用对准直接融合。\par
【贴出测试数据】\par
在融合之前,需要对图片进行预处理。步骤如下:
\begin{enumerate}
\item 使用附带的角度进行旋转。
\item 旋转后的图片进行裁剪,从而消除因旋转产生的黑边,并且展现需要融合图片的特征。
\end{enumerate}
预处理之后就可以进行融合了,融合之后的效果将会用熵值、标准差、空间频率,以及与原先照片相比较得出的互信息中位数来量化衡量。\par
【拿出 MSTAR 图像结果贴上去】
\section{仿真获取数据模块}
本模块用于描述本程序仿真SAR雷达扫描的步骤,以及对数据的处理。\par
这些代码都在terrain\_sar\_image\_simulator文件夹下。
\subsection{随机地形生成}
这段代码在`helperRandomTerrainGenerator.m`里面,描述了如何通过随机数等参数来生成模拟地形,用于接下来的扫描步骤。\par
[描写获取的参数]
[描写生成的步骤]
[贴图表示地形]
\subsection{模拟雷达参数}
【翻译Define an L-band SAR imaging system with a range resolution approximately equal to 5 m. This system is mounted on an airborne platform flying at an altitude of 1000 meters.根据此进一步分析】
\subsection{目标点放置}
【描述目标点】
\subsection{雷达运动轨迹}
参考了论文A中提到的 SAR COOPERATIVE OBSERVATION MODEL ,根据多角度成像的需求,我决定如此安排路径:

【描述根据目标点】
【绘制图画】
以下是程序运行过程中,所展示的路径概览。
【程序图片截图】
\subsection{模拟流程}
\subsection{数据保存}
【简介雷达数据】
【方向分辨率和速度分辨率的描述】
【成像一览】
\section{终端输入设计}
\subsection{选择图片}
\subsection{输入角度}
\subsection{进度条显示}
\chapter{运行展示}
\section{用户输入模式}
\section{仿真运行模式}

\chapter{总结和展望}
\end{document}